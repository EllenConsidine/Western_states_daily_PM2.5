\subsubsection{\texorpdfstring{PM\textsubscript{2.5}}{} Monitor data from Uintah Basin}

\subsubsection*{To Do}
\begin{enumerate}
\item Check to see if there is any more recent data available - email sent June 17, 2019
\end{enumerate}

\subsubsection*{Data Source}

\begin{itemize}[nolistsep]
\item \textbf{Contact} Seth Lyman 
\item \textbf{Citation/Link} seth.lyman@usu.edu
\item \textbf{Data (local)} PM\textsubscript{2.5} measurements from 10 sites in Uintah Basin, Utah
\item \textbf{Geographic Extent} Uintah Basin, Utah
\item \textbf{Temporal Extent} October 2009 - March 2017
\item \textbf{Acknowledgment} PM\textsubscript{2.5} data from the Uintah Basin were provided by Seth Lyman at Utah State University.
\end{itemize}

\subsubsection*{Brief Description}

PM\textsubscript{2.5} data were provided by Seth Lyman at Utah State University via email on January 16, 2018. The .xlsx file has PM\textsubscript{2.5} data from 10 stations during 2009-2017. The .png file has the longitude and latitude of each site. 

\subsubsection*{Notes}
Additional information from Seth's email: \newline
``I’ve attached most of the PM2.5 observations that have ever been collected in the Uintah Basin.  What are in the Excel file are 24-hr average data.  Data from Roosevelt, Vernal, Ouray, Red Wash, Myton, and Rangely are from the EPA AQS database. \newline Data from Horsepool are from a BAM 1020 monitor that we operate every winter.  Data in Ft. Duchesne and Randlett are 24-hr filter samples that were analyzed gravimetrically.  Data from Rabbit Mountain are from a BAM 1020, and data through mid-2013 are in the AQS database. \newline \newline
\noindent I have hourly data from Horsepool and Rabbit Mountain if you’d rather have that. \newline \newline
\noindent Site locations are given in the list of monitoring stations for the Basin below.'' \newline

The .png file is easier to read in some programs than others, e.g., it looks fine in ``Paint,'' but not ``Photos.''

\subsubsection*{File Formats} 
Excel and png

\subsubsection*{Data Filtering and Processing}
FinalPM2.5\_multiyear\_thruwint2017\_sheet1.csv is the first sheet of FinalPM2.5\_multiyear\_thruwint2017.xlsx converted to .csv, and the second row of the header was merged into the first (24hr avg PM2.5).

FinalPM2.5\_multiyear\_thruwint2017\_GISsheet.csv is the third sheet of FinalPM2.5\_multiyear\_thruwint2017.xlsx converted to .csv and gives the latitude and longitude of each site. This sheet originally did not have location information from the Rangely site, so this was filled in by hand with the numbers form UintahBasinSiteLocations.png.


\subsubsection*{Final Variable(s)}

\subsubsection*{Methods}

\begin{enumerate}
\item 
\item
\end{enumerate}

\subsubsection*{Quality Control}

\subsubsection*{Script Names}

\begin{enumerate}
\item 
\end{enumerate}

\subsubsection*{Original Data File Names}

\begin{enumerate}
\item FinalPM2.5\_multiyear\_thruwint2017.xlsx
\item UintahBasinSiteLocations.png
\end{enumerate}

\subsubsection*{Processed/Cleaned Data File Names}

\begin{enumerate}
\item FinalPM2.5\_multiyear\_thruwint2017\_sheet1.csv%FinalPM2.5\_multiyear\_thruwint2017_sheet1.csv
\item UintahBasinSiteLocations.png
\end{enumerate}