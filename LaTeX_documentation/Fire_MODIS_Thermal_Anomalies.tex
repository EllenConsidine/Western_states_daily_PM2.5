\subsection{MODIS Thermal Anomalies/Fire Daily L3 Global 1km (MCD14DL)}
\subsubsection*{Data Source}
\begin{itemize}[nolistsep]
\item \textbf{Contact}
\item \textbf{Citation/Link}
\item \textbf{Data (local)}
\item \textbf{Geographic Extent}
\item \textbf{Temporal Extent}
\item \textbf{Acknowledgment}
\end{itemize}
\subsubsection*{Brief Description}

We will collect data about fire detection locations, size, and fire radiative power from the MODIS Thermal Anomalies/Fire Daily L3 Global 1km (MOD14 and MYD14) \citep{Giglio2006,Hawbaker2017}. 
Using GIS techniques, we will create daily clusters of fire points and use these to calculate: (1) the distance to the nearest fire cluster by day and (2) the sum of Fire Radiative Power (FRP) of the nearest clusters of fires by day as it is likely that smoke levels are higher closer to fires. The MODIS product spans longer than our study period (2008-2014) at daily temporal resolution and has a spatial resolution of 1 km.

\subsubsection*{Notes}
\subsubsection*{File Format} .shp
\subsubsection*{Data Filtering and Processing}
\subsubsection*{Final Variable(s)}
\subsubsection*{Methods}
\begin{enumerate}
\item Navigate to the \href{https://firms.modaps.eosdis.nasa.gov/download/}{NASA EarthData FIRMS Archive Download site}
\item Select "Create new Request"
\item In the dropdown for region, select "Custom Region" and draw a bounding box around study area
\item In the dropdown for fire data source, select "MODIS C6"
\item Select dates for study time period
\item In the dropdown for file type, select "Shapefile (.shp)"
\item Enter your email address
\item You will get an email with a download link containing a zipfile with the data
\end{enumerate}
\subsubsection*{Quality Control}
\subsubsection*{Script Names}
\begin{enumerate}
\item n/a
\end{enumerate}
\subsubsection*{Data File Names}
\begin{enumerate}
\item n/a
\end{enumerate}