\subsection{\texorpdfstring{MAIAC AOD (MCD19A2)}}

\subsubsection*{Data Source}

\begin{itemize}[nolistsep]
\item \textbf{Contact} 
\item \textbf{Citation/Link} https://ladsweb.modaps.eosdis.nasa.gov/archive/allData/6/MCD19A2/?process=ftpAsHttp&path=allData%2f6%2fMCD19A2
\item \textbf{Data (local)} 
\item \textbf{Geographic Extent} 
\item \textbf{Temporal Extent} 
\item \textbf{Acknowledgment} 
\end{itemize}

\subsubsection*{Brief Description}
https://lpdaac.usgs.gov/dataset_discovery/modis/modis_products_table/mcd19a2_v006
"The MCD19A2 Version 6 data product is a MODIS Terra and Aqua combined Multi-angle Implementation of Atmospheric Correction (MAIAC) Land Aerosol Optical Depth (AOD) gridded Level-2 product produced daily at 1 kilometer (km) pixel resolution. The MCD19A2 product provides the atmospheric properties and view geometry used to calculate the MAIAC Land Surface Bidirectional Reflectance Factor (BRF), or surface reflectance, MCD19A1 product."


\subsubsection*{Notes}
Because the MAIAC is a Level2G (gridded) product, the lat-lon coordinates are not included in each HDF file (these files are sinusoidally-projected). Read more about the MODLAND sinusoidal projection here: https://modis-land.gsfc.nasa.gov/MODLAND_grid.html Thus, we must download the relevant tiles from ftp://dataportal.nccs.nasa.gov/DataRelease/MODISTile_lat-lon/ before beginning processing. (Note: this download is not automated because we would need a NASA login to call from the ftp programatically.)

Converting to shp and raster files was taking too long (because of the fine resolution), so we used a k-nearest-neighbors approach to estimate the AOD values at each monitor location. 

\subsubsection*{File Formats} 
HDF; data is in gridded file format

\subsubsection*{Data Filtering and Processing}

\subsubsection*{Final Variable(s)}
CSV file of monitor locations, dates and AOD values

\subsubsection*{Methods}

\begin{enumerate}
\item Download the relevant lat-lon tiles from ftp://dataportal.nccs.nasa.gov/DataRelease/MODISTile_lat-lon/
\item Download HDF files using download_from_https.py (in the MAIAC_AOD folder) 
\item Turn the HDF files into CSV files with average values for each day by running maiac_main.py, which calls maiac_create_csv.py
\item Extract a weighted average of AOD at each monitor location based on the nearest k points (k = user specified value) listed in the average-value CSV file for each day, using k_nearest_neighbors.py
\end{enumerate}

\subsubsection*{Quality Control}

\subsubsection*{Script Names}

\begin{enumerate}
\item download_from_https.py (in the MAIAC_AOD folder)
\item maiac_main.py (calls maiac_create_csv.py)
\item k_nearest_neighbors.py
\end{enumerate}

\subsubsection*{Original Data File Names}

\begin{enumerate}
\item 
\item 
\end{enumerate}

\subsubsection*{Processed/Cleaned Data File Names}

\begin{enumerate}
\item 
\item 
\end{enumerate}
