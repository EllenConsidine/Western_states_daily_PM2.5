\section{Introduction}

The increase in frequency and severity of wildfires occurring in the western US \citep{Dennison2014,Steel2014} has led to higher air pollution levels than would be expected without the fires \citep{odell_contribution_2019}. 
Prescribed fires (deliberate, controlled fires) are used as a management tool to reduce fuel loads and the risk of large uncontrolled wildfires while allowing ecological benefits of fire. Previous research indicates that prescribed fires impact air quality less than wildfires on a per-fire basis \citep{Liu2017}, however differences in chemical composition of the smoke could result in different health impacts from these two types of fires. To our knowledge, previous studies have not considered if air pollution from prescribed fires and wildfires pose differential risks to public health. Wildfire smoke is a complex mixture of gases and particles, but most studies of wildfire smoke and health focus on one air pollutant, particulate matter with aerodynamic diameter smaller than 2.5 $\mu \textrm{m}$ (PM\textsubscript{2.5}), which is the predominant pollutant of concern to health in wildfire smoke \citep{sapkota_impact_2005}. 

Recent extremely large fires such as the Camp Fire in Paradise, CA and the 2017 fires in Santa Rosa, CA have burned not only vegetation but also many buildings. We hypothesize that these urban-invasive fires could have different air pollution chemistry which could affect population health differently than wildfires that burn predominantly forests or grasses. Compared to our understanding of the health impacts of other sources of air pollution, we know much less about the health impacts of wildfire smoke exposure. Recent review articles document that wildfire smoke exacerbates asthma, although the evidence is less consistent for other respiratory and cardiovascular outcomes \citep{reid_wildfire_2018,reid_critical_2016, LIU2015120}, despite clearer associations found with other sources of air pollution \citep{Requia_global_2018}. It is possible that differences in findings across studies could be due to the type of fuels burned or other characteristics of each wildfire, or they could be due to the prevalence of chronic disease within the exposed populations \citep{LIU2015120,rappold_peat_2011,tinling_repeating_2016}. To date, however, no one has investigated what is driving heterogeneity of effects in wildfire-health studies. Given the increased risk of wildfires in the western US \citep{westerling_increasing_2016,schoennagel_adapt_2017}) and the increased population living within the wildland-urban interface \citep{radeloff_rapid_2018}, it is increasingly important to understand whether different kinds of wildfires pose differential health risks and if wildfires exert more health concerns in areas with higher pre-existing health burdens. As a pre-requisite to such a study, this paper presents the exposure data necessary.

Our interdisciplinary team has experience modeling fire emissions and trajectories of smoke plumes \citep{thelen_modeling_2013}, estimating spatiotemporal air pollution exposures \citep{thelen_modeling_2013,reid_spatiotemporal_2015}, and analyzing epidemiological relationships between air pollution and health \citep{reid_differential_2016,crooks_association_2016}. 
Our proposal addresses the following \textbf{Specific Aims}: 
\medskip

\noindent\textbf{AIM 1}: Determine whether there are differential associations between PM\textsubscript{2.5} and respiratory hospitalizations and emergency department (ED) visits during prescribed fires compared to wildfires.

\textbf{Aim 1a} – For a list of 200 wildfires and 150 prescribed fires that occurred in our study area (AZ, CA, CO, NM, OR, UT, WA) during 2008-2018, estimate the background and fire-affected PM\textsubscript{2.5} concentrations for each affected ZIP code-day.

\textbf{Aim 1b} – Using Poisson generalized estimating equations (GEE) models assess associations between prescribed fire PM\textsubscript{2.5} and wildfire PM\textsubscript{2.5}, separately, with respiratory hospitalizations and ED visits, adjusted for pertinent confounding variables. 

\noindent\textbf{AIM 2}: Assess whether associations between PM\textsubscript{2.5} during wildfires and respiratory and cardiovascular hospitalizations and ED visits differ by wildfire characteristics (e.g., predominant fuel type, size, duration, etc.).

\textbf{Aim 2a} – For each fire, estimate 
the association between daily wildfire-affected PM\textsubscript{2.5} and daily counts of respiratory and cardiovascular hospitalizations and ED visits using Poisson GEE models, adjusted for pertinent confounding variables. 

\textbf{Aim 2b} – Using meta-regression of the individual regressions by fire from Aim 2a, quantify the role of fire characteristics in describing the heterogeneity in relationships between PM\textsubscript{2.5} and respiratory and cardiovascular hospitalizations and ED visits across the study fires. 

\noindent\textbf{AIM 3}: Assess whether associations between PM\textsubscript{2.5} during wildfires and respiratory and cardiovascular hospitalizations and ED visits differ by population health characteristics (e.g., prevalence of chronic disease or health behaviors).

\textbf{Aim 3a} – Using meta-regression of the individual regressions by fire from Aim 2a, quantify the role of population health characteristics in describing the heterogeneity in relationships between PM\textsubscript{2.5} and respiratory and cardiovascular hospitalizations and ED visits across the study fires.