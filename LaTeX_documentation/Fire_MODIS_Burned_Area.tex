\subsection{MODIS/Terra and Aqua Burned Area Monthly L3 Global 500 m SIN Grid V006 (MCD64A1) }
\subsubsection*{Data Source}
\begin{itemize}[nolistsep]
\item \textbf{Contact}
\item \textbf{Citation/Link}
\item \textbf{Data (local)}
\item \textbf{Geographic Extent}
\item \textbf{Temporal Extent}
\item \textbf{Acknowledgment}
\end{itemize}
\subsubsection*{Brief Description}

We will collect data about fire detection locations, size, and fire radiative power from MODIS/Terra and Aqua Burned Area Monthly L3 Global 500 m SIN Grid V006 (MCD64A1) 
\citep{Schroeder2014}. % check if this is the right citation
Using GIS techniques, we will create daily clusters of fire points and use these to calculate: (1) the distance to the nearest fire cluster by day and (2) the sum of Fire Radiative Power (FRP) of the nearest clusters of fires by day as it is likely that smoke levels are higher closer to fires.

\subsubsection*{Notes}
\subsubsection*{File Format} .hdf
\subsubsection*{Data Filtering and Processing}
\subsubsection*{Final Variable(s)}
\subsubsection*{Methods}
\begin{enumerate}
\item Run script `Generic_FTP_download_to_local.py` and pass two arguments: the first is the data set name and the second is the local directory path to save files to (i.e. "MCD64A1" "C:/Users/User/MCD64A1\_Downloads")
\end{enumerate}
\subsubsection*{Quality Control}
\subsubsection*{Script Names}
\begin{enumerate}
\item MODIS\_FTP\_Download.py
\end{enumerate}
\subsubsection*{Data File Names}
\begin{enumerate}
\item 
\end{enumerate}