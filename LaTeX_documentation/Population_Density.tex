\subsection{\texorpdfstring{Population Density}}

\subsubsection*{Data Source}

\begin{itemize}[nolistsep]
\item \textbf{Contact} 
\item \textbf{Citation/Link} https://www.socialexplorer.com/tables/C2010
\item \textbf{Data (local)} 
\item \textbf{Geographic Extent} 
\item \textbf{Temporal Extent} 
\item \textbf{Acknowledgment} 
\end{itemize}

\subsubsection*{Brief Description}
Population density at the Census tract level; data from the 2010 US Census. 

\subsubsection*{Notes}
We obtained a Census tract shapefile from ESRI. Any Census tract shapefile covering the western US will work, as long as it includes FIPS codes that may be matched with the population density data.

\subsubsection*{File Formats} 
Shapefile, CSV

\subsubsection*{Data Filtering and Processing}

\subsubsection*{Final Variable(s)}

\subsubsection*{Methods}

\begin{enumerate}
\item Merge open source population density data with a Census tract shapefile, creating a spatial polygons object.
\item Intersect PM2.5 monitor locations with the spatial polygons object. 
\end{enumerate}

\subsubsection*{Quality Control}

\subsubsection*{Script Names}

\begin{enumerate}
\item Get_population_density.R
\end{enumerate}

\subsubsection*{Original Data File Names}

\begin{enumerate}
\item 
\item 
\end{enumerate}

\subsubsection*{Processed/Cleaned Data File Names}

\begin{enumerate}
\item 
\item 
\end{enumerate}
