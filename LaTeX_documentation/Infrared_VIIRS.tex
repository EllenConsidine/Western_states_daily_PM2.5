\subsection{Visible Infrared Imaging Radiometer Suite (VIIRS) (VNP14IMGTDL\_NRT) }
\subsubsection*{Data Source}
\begin{itemize}[nolistsep]
\item \textbf{Contact}
\item \textbf{Citation/Link}
\item \textbf{Data (local)}
\item \textbf{Geographic Extent}
\item \textbf{Temporal Extent}
\item \textbf{Acknowledgment}
\end{itemize}
\subsubsection*{Brief Description}

We will collect data about fire detection locations, size, and fire radiative power from the Visible Infrared Imaging Radiometer Suite (VIIRS) (VNP14IMGTDL\_NRT) 
\citep{Schroeder2014}. % not sure if that's the right citation
Using GIS techniques, we will create daily clusters of fire points and use these to calculate: (1) the distance to the nearest fire cluster by day and (2) the sum of Fire Radiative Power (FRP) of the nearest clusters of fires by day as it is likely that smoke levels are higher closer to fires. The MODIS product spans longer than our study period (2008-2014) at daily temporal resolution and has a spatial resolution of 1 km. VIIRS was launched in 2011 and has 12 h temporal resolution with 750 m resolution. The BAECV can detect fires larger than 4 km\textsuperscript{2} and provides an estimate of the date of the fire and is available from 1984-2015. 

\subsubsection*{Notes}
\subsubsection*{File Format} .csv
\subsubsection*{Data Filtering and Processing}
\subsubsection*{Final Variable(s)}
\subsubsection*{Methods}
\begin{enumerate}
\item Navigate to the \href{https://firms.modaps.eosdis.nasa.gov/download/}{NASA EarthData FIRMS Archive Download site}
\item Select "Create new Request"
\item In the dropdown for region, select "Custom Region" and draw a bounding box around study area
\item In the dropdown for fire data source, select "VIIRS"
\item Select dates for study time period
\item In the dropdown for file type, select "Shapefile (.shp)"
\item Enter your email address
\item You will get an email with a download link containing a zipfile with the data
\item Progress stopped here, as we chose to proceed with the MODIS Thermal Anomalies dataset for the active fire input for the project as of Fall 2018. But, follow along in the steps for the MODIS Thermal Anomalies workflow to continue. The steps are the same, as the data comes from the same source.
\end{enumerate}
\subsubsection*{Quality Control}
\subsubsection*{Script Names}
\begin{enumerate}
\item n/a
\end{enumerate}
\subsubsection*{Data File Names}
\begin{enumerate}
\item fire\_archive\_V1\_2770.csv
\end{enumerate}
