\section{Compiling Data}

\subsection{Processing PM2.5 data}

These are the scripts that process and compile the PM2.5 data:
\begin{enumerate}[nolistsep]
\item \textul{Script1\_Install\_Pkgs.R} >> install packages

\item \textul{Define\_directories.R} >> clears all variables and defines directories. Needs to be ran between each of the following scripts. (Want to automate this eventually.)

\item % \textul{Create\_ML\_Input\_File.R} 
\textul{Process\_PM25\_data\_step1.R} >> compiles the various PM2.5 data sources into a single data frame. The only eliminations of data are geographic, to remove states that are neither in our study area nor adjacent to it.
	\begin{enumerate}
	\item For DRI data, put in flags for voltage data outside the range 11-17 V. (These thresholds are somewhat arbitrary, but it was noticed that when the voltage was outside this range, the PM\textsubscript{2.5} concentrations were often absurdly high, e.g., greater than 24,000 ug/m3.
	\end{enumerate}

%\item \textul{Clean\_ML\_Input\_File.R} >> cleans the data. The following is a list of the quality cuts made on the data:clean data, e.g., negative concentrations
\item \textul{Process\_PM25\_data\_step2.R} >> cleans the data. The following is a list of the quality cuts made on the data:clean data, e.g., negative concentrations
	\begin{enumerate}[nolistsep]
	\item negative and NA concentrations (this includes removing all data for a monitor on a given day if any of the hourly observations were negative)
	\item For the hourly data, remove monitor-days that do not have at least 18/24 observations
	\item For DRI data, remove data with voltage flags (which includes flags that came with the data and flags that were put in because the battery voltage was outside the range 11-17 V.
	\item June 6, 2014 24-hr average PM\textsubscript{2.5} concentration from monitor ``Smoke NCFS E-BAM \#1'' (Fire\_Cache\_Smoke\_DRI\_Smoke\_NCFS\_E\_BAM\_N1.csv) is 24,203 ug/m3. There's nothing apparent wrong with the hourly data, however, this is the only day of data that made it through the other quality checks from this data file. This suggests that this monitor is suspect, and will be removed.
	\item Remove data points with lat/lon outside this box: (50,-126) to (25,-93) 
	\item \textbf{To Do} make cuts on air flow in DRI data - at least get rid of negative air flow and think about tighter thresholds
	\item \textbf{To Do} think about making cuts on any unrealistic air temperatures for DRI data
	\item \textbf{To Do} why are some of the Site\_Num values not integers?
	\item \textbf{To Do} why is there a longitude value of -349?
	\item \textbf{To Do} need to convert missing values that have a -9999 etc to NA value
	\item \textbf{To Do} figure out why some latitudes have a negative value
	\item \textbf{To Do} figure out why PM2.5 Lon has value of 0 sometimes
	\item \textbf{To Do} merge "24-HR BLK AVG" and "24 HOUR" data together in Sample Duration variable
	\item \textbf{To Do} figure out why Observation percent has a max value of 200\%
	\item \textbf{To Do} figure out why PM2.5 Obs has a max value of 349000 ug/m3 and remove it
	\item \textbf{To Do} remove unrealistic PM2.5 data values
	\item \textbf{To Do} figure out if max AQI value of 546 is reasonable
	\item \textbf{To Do} remove data from after 2014
	\item \textbf{To Do} Some DRI files looked like they had hour 20:00 data shifted a couple of columns - look into this and fix it.
	\item \textbf{To Do} look over summary() output and plots of every variable and determine if any other cuts are necessary

	\end{enumerate}

\textul{Process\_PM25\_data\_step3.R} >> convert all PM2.5 data to the same datum (NAD83) and project coordinates. Take the reprojected location info and put it into the data frame with the daily PM\textsubscript{2.5} data.

\textul{Process\_PM25\_data\_step4.R} >> combine locations and location/date files for parts b and c. (Part a should be disregarded.)%separate part a locations from part b, etc - to be used for extracting predictor variables to points of monitor locations and points of interest for predicting PM2.5

%\item \textul{write\_shp.py} given listing of all PM2.5 monitor locations (in different datums), write each datum into a different shapefile
%\item \textul{reproject\_shp.py} project all shapefiles into US Equal Area Albers (ESRI 102003); save coordinates
%\item \textul{reproject\_shp\_2.py} transform all original shapefiles into NAD83 datum (ESRI 4269); save coordinates 
%\item \textul{merge\_shp.py} merge all the shapefiles into one
%\item \textul{join\_coordinates.py} combine the geographic coordinates and projection coordinates and write to one CSV

%\item \textul{Process\_PM25\_data\_step5.R} >> script to take the reprojected location info and put it into the data frame with the daily PM\textsubscript{2.5} data. %\textul{Merge\_reprojections\_ML\_Input\_File.R}

\item \textul{Process\_PM25\_data\_step5.R} >> composite replicate data - in process. Calls these functions:
  \begin{enumerate}
  \item Combine\_true\_replicates\_R\_function.R
  \item fill\_in\_aves\_coloc\_unique\_PC\_POC\_MN\_function.R
    \begin{enumerate}
    \item concatinate\_within\_column\_function.R
    \end{enumerate}
  \item set\_data\_types\_by\_column\_R\_function.R
  \end{enumerate}
\item \textul{Plot\_ML\_Input\_File.R} >> create plots, maps, and statistical summary - needs to be changed to take input from De-duplicate code instead of file from Create\_ML\_Input\_File.R
\item \textbf{to be written} >> merge with satelite and other data
\end{enumerate}
% {DeDuplicate\_ML\_Input\_File\_v2.R}

\subsubsection{Notes about very high data points}

June 15, 2012 24-hr average PM\textsubscript{2.5} concentration from monitor ``Smoke \#22'' (Fire\_Cache\_Smoke\_DRI\_Smoke\_22.csv) is 5,638 ug/m3 - can't find any reason, so far, to remove this data point, though it's very odd that the concentrations were low single-digits except for two hours which were extremely high (123,000 and 1000 ug/m3).



