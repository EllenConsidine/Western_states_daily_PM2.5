%\section{Machine Learning Methods}

\subsection{ML Techniques and Calculations}

Need to describe how R\textsuperscript{2} is calculated.

setting aside a portion of the PM2.5 data set and then doing 10-fold cross validation on the rest of the data

see \url{http://www.cvent.com/events/nasa-aist-machine-learning-workshop/event-summary-1f5144a5d1734ca39485d999bcdfc54a.aspx} and particularly the very end of \url{https://global.gotomeeting.com/public/recording-player.html?id=owZDmUustOjaW9sJGQ5u9cUG2pBa4D} for list of resources and papers to read.


\subsection{ML Scripts}

\begin{enumerate}

\item ML\_PM25\_estimation\_merge\_predictors.R >>  Merge the various predictor variables together with the monitor data or dates/locations of interest % %ML\_PM25\_estimation\_step0.R

\item ML\_PM25\_estimation\_plot\_predictors.R >> Plot the training input file % ML\_PM25\_estimation\_step0a.R
	\begin{enumerate}
		\item predictor variables vs date
		\item predictor variables vs PM\textsubscript{2.5}
	\end{enumerate}

\item ML\_PM25\_estimation\_step1.R >> ML training algorithms

\item ML\_PM25\_estimation\_step1.R >> create data frame of the dates/locations for which we want to predict PM2.5

\end{enumerate}