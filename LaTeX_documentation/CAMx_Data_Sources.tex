\section{Data Sources for CAMx Modeling of Source-Attributed Air Quality Modeling}\label{sec:CAMxDataSources}

For meteorological inputs, the CAMx modeling will use archived daily 27-km Advanced Research Weather Research and Forecasting (WRF-ARW) grids available via NOAA Real-time Environmental Applications and Display sYstem (READY) servers for the entire study area and time period \citep{Wang2007,Rolph2017}. For the study years 2008-2012 and 2014, we will use fire emissions datasets prepared by the Western Regional Air Partnership (WRAP) and the National Emissions Inventory (NEI) \citep{EPANEI2017} based on aggregated source-tagged fire occurrence data sources, the FCCS \citep{Ottmar2007}, and Consume \citep{Prichard2009}
modeling. For the study year 2013, we will prepare a fire emissions dataset using the same aggregated source-tagged fire occurrence data sources and FCCS/Consume modeling framework in the NASA-funded  Wildland Fire Emissions Information System (WFEIS) \citep{WFEIS2017} developed by Co-I's French and Billmire \citep{French2014}. Fire occurrence datasets include MODIS  (MOD14/MYD14 and MCD64A1) and  VIIRS (VNP14IMGTDL\_NRT) fire data products \citep{Giglio2006,MODISBurnArea,Schroeder2014}. 
For non-fire emissions during the entire study period, we will use the dataset prepared by WRAP for year 2008.

Look into using spot forecasts to help distinguish between wild and prescribed fires: \url{http://www.weather.gov/spot/monitor/}