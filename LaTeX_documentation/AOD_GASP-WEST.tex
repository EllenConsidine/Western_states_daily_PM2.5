\subsection{GASP-West AOD}

\subsubsection*{Data Source}

\begin{itemize}[nolistsep]
\item \textbf{Contact}
\item \textbf{Citation/Link}
\item \textbf{Data (local)}
\item \textbf{Geographic Extent}
\item \textbf{Temporal Extent}
\item \textbf{Acknowledgment}
\end{itemize}

\subsubsection*{Brief Description}

We will use AOD estimates from the Geostationary Operational Environmental Satellite West (GOES-West) Aerosol Smoke Product (GASP-West AOD). The GASP product is available at a 4 km resolution at nadir with retrievals every 30 minutes during daylight hours and is available from 2006 onward 
\citep{GASPAerosolProduct2017}.  

AOD products use cloud filtering algorithms that often remove pixels in the center of the smoke plumes because they are assumed to be clouds due to high reflectivity \citep{kondragunta_revisions_2009}. Given that these can be in the middle of smoke plumes, often the locations most heavily impacted by smoke have missing data for a key variable, AOD. In our previous work in summer in California when rain clouds are incredibly rare, we could be confident that missing values not along the coast were not clouds. However, for this larger study region and time period, this will be a bigger challenge. We will attempt to isolate smoke plumes from true clouds using satellite imagery and smoke plume polygons from NOAA's Hazard Mapping System Fire Smoke Product  \citep{NOAAHazMap2017}. We will then estimate missing values within validated smoke plumes, but not within clouds, using radial basis functions as was done in our previous work \citep{Reid2015}. Radial basis functions are exact interpolation functions that will return observed AOD values where they exist but can interpolate higher values than nearby observations in missing locations, which is needed since the missing values were removed due to their high reflectivity \citep{Reid2015}.

\subsubsection*{Notes}

\subsubsection*{File Format}

\subsubsection*{Data Filtering and Processing}

\subsubsection*{Final Variable(s)}

\subsubsection*{Methods}

\begin{enumerate}
\item Navigate to NCEI's \href{https://www.ncdc.noaa.gov/has/has.dsselect}{Archive Information Request System (AIRS)}. Scroll down and click on 'Satellite' to expand menu. Click on 'Goes Products' to expand menu. Click on 'Order Data'.
\item Select appropriate Satellite ID for time frame of interest (we selected GOES-11 for 01/01/2008-02/13/2012 and GOES-15 for 02/14/2012-12/31/2014 to encompass our study time period of 2008-2014). Select appropriate data type (GASP-AOD-GZ-).
\item Select "Yes" for Submit Batch
\item Enter email address and submit order
\end{enumerate}

\subsubsection*{Quality Control}

\subsubsection*{Script Names}

\begin{enumerate}
\item 
\end{enumerate}

\subsubsection*{Data File Names}

\begin{enumerate}
\item 
\end{enumerate}
