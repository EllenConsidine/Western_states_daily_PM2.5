\subsection{Classified land cover information from the Landsat-derived NLCD 2011}
\subsubsection*{Data Source}
\begin{itemize}[nolistsep]
\item \textbf{Contact}
\item \textbf{Citation/Link}
\item \textbf{Data (local)}
\item \textbf{Geographic Extent}
\item \textbf{Temporal Extent}
\item \textbf{Acknowledgment}
\end{itemize}
\subsubsection*{Brief Description}

Classified land cover information from the Landsat-derived NLCD 2011 
\citep{Homer2017} will be used to calculate estimates of the percentage of urban development (codes 22, 23, and 24), agriculture (codes 81 and 82), and vegetated area other than agricultural land (codes 21, 41, 42, 43, 52, and 71) within buffer radii of 100 m, 250 m, 500 m, and 1000 m around each monitor. The buffer distance that is most highly correlated with PM\textsubscript{2.5} will be entered into each model. NLCD 2011 has a spatial resolution of 30 m and uses circa 2011 Landsat satellite data. 

\subsubsection*{Notes}
\subsubsection*{File Format} .shp
\subsubsection*{Data Filtering and Processing}
\subsubsection*{Final Variable(s)}
\subsubsection*{Methods}
\begin{enumerate}
\item Navigate to the \href{https://viewer.nationalmap.gov/basic/}{National Map Viewer} and find products for "National Land Cover Database (NLCD)" at the National extent. From the search results, download "NLCD 2011 Land Cover (2011 Edition, amended 2014)". This will download a zipfile with the data.
\item Run nlcd\_process.py with the required arguments. This script computes zonal statistics between a buffer shp file and an classified raster tif (in our use case, a reclassified NLCD raster). The computed value is percent area of developed high density land cover in each buffer. The output is another csv, which is the input csv with an an extra column denoting the data.
Note: the buffer shp file used in this study consisted of 1km, 5km, and 10km radius buffers using planar buffering.
\end{enumerate}
\subsubsection*{Quality Control}
\subsubsection*{Script Names}
\begin{enumerate}
\item nlcd\_process.py
\end{enumerate}
\subsubsection*{Data File Names}
\begin{enumerate}
\item n/a
\end{enumerate}