\section{Ideas, To Do, Resources, etc}

Consider using the work of Westerling et al for a comprehensive fire history (up through 2012) \url{http://science.sciencemag.org/content/313/5789/940}, \url{http://www.pnas.org/content/108/32/13165}, \url{http://rstb.royalsocietypublishing.org/content/371/1696/20150178} \cite{westerling_increasing_2016,WesterlingCorrection2016} Also look into the fire histories referenced in Westerling \cite{westerling_increasing_2016,WesterlingCorrection2016}: \url{http://fam.nwcg.gov/fam-web/weatherfirecd/fire_files.htm} and \url{http://fam.nwcg.gov/fam-web/kcfast/mnmenu.htm} See also \url{http://www.nifc.gov}

Look at \cite{kollanus_effects_2016} again for references for PM2.5 paper, especially the introduction. Consider using NAAPS in our study. 

Idea: look at ambulance calls and PM2.5, similar to what \cite{salimi_ambient_2016} did in Australia.

US National Atlas \url{http://nationalmap.gov/small_scale/atlasftp.html}

Thought: Using DigitalGlobe for fire data compared to NASA: would have higher spatial resolution, but not consistently viewing all areas (no cost to CU people) 
% Sentinal only goes back a couple of years

%Look

According to \cite{liu_particulate_2016}, GEOS-Chem ``can be classified according to emission source'', that implies that we could tag the emissions as wildfire vs prescribed fire vs urban. Would there be any advantages of this model over CAMx?

could analyze data with NAAQS and WHO PM2.5 standards

projection info: \url{https://gis.stackexchange.com/questions/664/whats-the-difference-between-a-projection-and-a-datum}
\url{http://resources.esri.com/help/9.3/arcgisengine/dotnet/89b720a5-7339-44b0-8b58-0f5bf2843393.htm}