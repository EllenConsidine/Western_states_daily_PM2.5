\section{Ideas, To Do, Resources, etc}

code up fires by type of land coverage

Consider using the work of Westerling et al for a comprehensive fire history (up through 2012) \url{http://science.sciencemag.org/content/313/5789/940}, \url{http://www.pnas.org/content/108/32/13165}, \url{http://rstb.royalsocietypublishing.org/content/371/1696/20150178} \cite{westerling_increasing_2016,WesterlingCorrection2016} Also look into the fire histories referenced in Westerling \cite{westerling_increasing_2016,WesterlingCorrection2016}: \url{http://fam.nwcg.gov/fam-web/weatherfirecd/fire_files.htm} and \url{http://fam.nwcg.gov/fam-web/kcfast/mnmenu.htm} See also \url{http://www.nifc.gov}

look into the Fire and Smoke Model Evaluation Experiment (FASMEE) \url{http://www.fasmee.net}

Compare our results with EPA Downscaler \url{https://www.epa.gov/air-research/downscaler-model-predicting-daily-air-pollution}

Look at \cite{kollanus_effects_2016} again for references for PM2.5 paper, especially the introduction. Consider using NAAPS in our study. 

read \cite{mcclure_US_2018}

read \cite{landis_impact_2018}

see also \url{https://www.5280.com/2018/09/can-colorado-burn-its-way-out-of-a-wildfire-crisis/}

Could we use inciweb to distinguish prescribed fires?

look up Global Fire Emissions Database (GFED3) - maybe it would be useful for our study as an input to the machine learning? see \cite{liu_wildfire-specific_2016}

see \cite{alman_association_2016} for potential data sources for ML project

emissions vary by temperature \url{https://cires.colorado.edu/news/wildfire-temperatures-key-better-understanding-air-quality} and \url{https://www.atmos-chem-phys.net/18/9263/2018/}

read Monitoring Trends in Burn Severity MTBS, 2014. Data Access: Fire Level Geospatial
Data. US Department of Agriculture, Forest Service and US Department of
Interior, Geological Survey. \url{http://mtbs.gov/data/individualfiredata.html/}.

Idea: look at ambulance calls and PM2.5, similar to what \cite{salimi_ambient_2016} did in Australia.

read \cite{williamson_transdisciplinary_2016}

Database of planned/proposed prescribed burns: WRAP's Fire Emissions Tracking System: \url{http://wrapfets.org/index.cfm}

See Di et al., 2016 and Johnston et al., 2012, Rappold et al., 2014 in \cite{jones_application_2017} - combine modelled and monitored/satelite data to estimate PM2.5

See page 11 of \cite{hyde_air_2017} for discussion of discrepancies related to burned area estimates

\url{http://www.ptep-online.com/ctan/symbols-a4.pdf}

US National Atlas \url{http://nationalmap.gov/small_scale/atlasftp.html}

Thought: Using DigitalGlobe for fire data compared to NASA: would have higher spatial resolution, but not consistently viewing all areas (no cost to CU people) 
% Sentinal only goes back a couple of years

Look up Openair R package

Papers/resources to look into: \url{https://daac.ornl.gov/cgi-bin/dsviewer.pl?ds_id=1293}

\url{https://www.fs.fed.us/psw/publications/4451/psw_2009_4451-001.pdf}

\url{https://labcit.ligo.caltech.edu/~ethrane/Resources/UNIX/}

\url{https://community.tableau.com/thread/141548}

According to \cite{liu_particulate_2016}, GEOS-Chem ``can be classified according to emission source'', that implies that we could tag the emissions as wildfire vs prescribed fire vs urban. Would there be any advantages of this model over CAMx?

could analyze data with NAAQS and WHO PM2.5 standards

projection/datum info: \url{https://gis.stackexchange.com/questions/664/whats-the-difference-between-a-projection-and-a-datum}
\url{http://resources.esri.com/help/9.3/arcgisengine/dotnet/89b720a5-7339-44b0-8b58-0f5bf2843393.htm}
\url{http://grindgis.com/blog/wgs84-vs-nad83}

Monitoring Trends in Burn Severity (MTBS) MTBS, 2016: Data Access: Fire Level Geospatial Data. USDA Forest Service/U.S. Geological Survey, accessed 8 October 2016, https://mtbs.gov/direct-download. 
Eidenshink, J., B. Schwind, K. Brewer, Z.-L. Zhu, B. Quayle, and S. Howard, 2007: A project for monitoring trends in burn severity. Fire Ecol., 3, 3–21, https://doi.org/10.4996/fireecology.0301003. 

Idea: Maybe instead of just distance to closest fire, we should follow the example of [Baek2016] and do distributed lags with concentric circles with information about fires in each concentric circle... also, instead of just distance to fire, maybe we could come up with a variable that is something like [distance*size of fire] since both are important.

Fire stats/records: \url{https://www.nifc.gov/fireInfo/fireInfo_statistics.html}

See \cite{adelaine_assessment_2017} for description of fire perimeter data that perhaps we could use (CA only)

See \cite{vaidyanathan_developing_2018} for info about MTBS and Active Fire Mapping Program and NWS smoke products. See also Lassman et al \cite{lassman_spatial_2017} cited therein.

Read these papers cited in \cite{lassman_spatial_2017}: Yao and Henderson, 2014; Henderson et al 2011; Liu et al 2015; Gan et al 2017; and look at their sources of PM2.5 data to see if we could add any of those to our project.

\section{PM2.5 Surface Paper Notes}

Discussion of trends in anthro PM2.5: \cite{ridley_causes_2018}

\subsection{Papers published in Atmospheric Environment - use as style example}

Need to go through these papers
\begin{itemize}
\item \cite{BrokampExposure2017} (partially done, done through intro)
\item \cite{Sampson2013}
\item \cite{Anyenda2016}
\item \cite{Torvela2014}
\item \cite{Whiteman2014}
\end{itemize}

Put in \cite{BrokampExposure2017,larsen_impacts_2017}

\section{Papers to cite/discuss in Introduction and/or Discussion}

\cite{westerling_increasing_2016,WesterlingCorrection2016}

try to find English version \url{http://80.24.165.149/webproduccion/PDFs/15CAP03.PDF}

For fire identification, consider using NOAA's Hazard Mapping System and BlueSky

\subsection{Notes on Papers}

See \cite{Fusco2016} for statistics about wildfires in western US, e.g., \% started by humans, number of fires, etc.

\section{Fire attribution paper}

revisit \cite{schweizer_using_2017}

include \cite{long_aligning_2018} - does a good job of summarizing the debate about more vs less prescribed burns

sources of fire data \cite{http://www.nifc.gov/fireInfo/fireInfo_main.html}, \cite{https://fam.nwcg.gov/fam-web/}

will need to compare our work to \cite{fann_health_2017}

include \cite{westerling_increasing_2016,WesterlingCorrection2016} and \cite{abatzoglou_impact_2016}

See \cite{Kaulfus2017} for an alternative method of attributing PM2.5 to wildfire smoke (instead of CAMx)

See Le et al 2014 \cite{http://dx.doi.org/10.3390/ijgi3020713}

See Huff et al \cite{http://dx.doi.org/10.4137/EHI.S19590}

\subsection{text written for the COPD paper - variation of this may be useful}
Larsen et al., 2017 \cite{larsen_impacts_2017} found that, on average, ground-level PM\textsubscript{2.5} concentrations increased by 2.9 \begin{math}\mu\text{g\textperiodcentered m}^{-3}\end{math} (2.8, 3.0) when there was a visible wildfire smoke plume overhead (from satellite imagery), as well as a 2.6 ppb (2.5-2.7) increase in O\textsubscript{3}. Satellite data provides a wealth of data and can provide information about air quality where monitors are not present. However, satellite imagery inherently comes with a substantial uncertainty in that satellite data describes the entire atmospheric column and not specifically just air pollution at the ground level, where people are breathing. % good points for our other paper, but may not apply here.
