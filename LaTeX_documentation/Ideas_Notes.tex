\section{Ideas, To Do, Resources, etc}

Consider using the work of Westerling et al for a comprehensive fire history (up through 2012) \url{http://science.sciencemag.org/content/313/5789/940}, \url{http://www.pnas.org/content/108/32/13165}, \url{http://rstb.royalsocietypublishing.org/content/371/1696/20150178} \cite{westerling_increasing_2016,WesterlingCorrection2016} Also look into the fire histories referenced in Westerling \cite{westerling_increasing_2016,WesterlingCorrection2016}: \url{http://fam.nwcg.gov/fam-web/weatherfirecd/fire_files.htm} and \url{http://fam.nwcg.gov/fam-web/kcfast/mnmenu.htm} See also \url{http://www.nifc.gov}

Look at \cite{kollanus_effects_2016} again for references for PM2.5 paper, especially the introduction. Consider using NAAPS in our study. 

Idea: look at ambulance calls and PM2.5, similar to what \cite{salimi_ambient_2016} did in Australia.

US National Atlas \url{http://nationalmap.gov/small_scale/atlasftp.html}

Thought: Using DigitalGlobe for fire data compared to NASA: would have higher spatial resolution, but not consistently viewing all areas (no cost to CU people) 
% Sentinal only goes back a couple of years

%Look

Papers/resources to look into: \url{https://daac.ornl.gov/cgi-bin/dsviewer.pl?ds_id=1293}

According to \cite{liu_particulate_2016}, GEOS-Chem ``can be classified according to emission source'', that implies that we could tag the emissions as wildfire vs prescribed fire vs urban. Would there be any advantages of this model over CAMx?

could analyze data with NAAQS and WHO PM2.5 standards

projection/datum info: \url{https://gis.stackexchange.com/questions/664/whats-the-difference-between-a-projection-and-a-datum}
\url{http://resources.esri.com/help/9.3/arcgisengine/dotnet/89b720a5-7339-44b0-8b58-0f5bf2843393.htm}
\url{http://grindgis.com/blog/wgs84-vs-nad83}

Monitoring Trends in Burn Severity (MTBS) MTBS, 2016: Data Access: Fire Level Geospatial Data. USDA Forest Service/U.S. Geological Survey, accessed 8 October 2016, https://mtbs.gov/direct-download. 
Eidenshink, J., B. Schwind, K. Brewer, Z.-L. Zhu, B. Quayle, and S. Howard, 2007: A project for monitoring trends in burn severity. Fire Ecol., 3, 3–21, https://doi.org/10.4996/fireecology.0301003. 

Idea: Maybe instead of just distance to closest fire, we should follow the example of [Baek2016] and do distributed lags with concentric circles with information about fires in each concentric circle... also, instead of just distance to fire, maybe we could come up with a variable that is something like [distance*size of fire] since both are important.

Fire stats/records: \url{https://www.nifc.gov/fireInfo/fireInfo_statistics.html}

\section{PM2.5 Surface Paper Notes}

\subsection{Papers published in Atmospheric Environment - use as style example}

Need to go through these papers
\begin{itemize}
\item \cite{BrokampExposure2017} (partially done, done through intro)
\item \cite{Sampson2013}
\item \cite{Anyenda2016}
\item \cite{Torvela2014}
\item \cite{Whiteman2014}
\end{itemize}

Put in \cite{BrokampExposure2017,larsen_impacts_2017}

\section{Papers to cite/discuss in Introduction and/or Discussion}

\cite{westerling_increasing_2016,WesterlingCorrection2016}

\subsection{Notes on Papers}

See \cite{Fusco2016} for statistics about wildfires in western US, e.g., \% started by humans, number of fires, etc.
