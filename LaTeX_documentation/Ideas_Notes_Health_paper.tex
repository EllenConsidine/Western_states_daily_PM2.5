\section{Ideas and Notes for paper}

See this paper for discussion of changes in toxicity of smoke between smoldering/open flame: Kim YH, Tong H, Daniels M, Boykin E, Krantz QT, McGee J, et al.
Cardiopulmonary toxicity of peat wildfire particulate matter and the predictive
utility of precision cut lung slices. Part Fibre Toxicol. 2014;11(1):1–17.

Discuss errors/uncertainties in assigning exposure data in health studies: (references in Linares et al., 2018 \cite{linares_impact_2018}
\begin{enumerate}
\item Weichenthal, S., Kulka, R., Lavigne, E., van Rijswijk, D., Brauer,M., Villeneuve, P.J., Stieb, D.,
Joseph, L., Burnett, R.T., 2017. Biomass burning as a source of ambient fine particulate
air pollution and acute myocardial infarction. Epidemiology 28 (3), 329–337 (May). \url{https://insights.ovid.com/crossref?an=00001648-201705000-00005}

\item Ingebrigtsen, R., Steinsland, I., Cirera, L1., Saez, M., 2015. Spatially misaligned data and the
impact of monitoring network on health effect estimates (Doctoral thesis at NTNU).
In: Ingebrigtsen, R. (Ed.), Bayesian Spatial Modelling of Non-stationary Processes
and Misaligned Data Utilising Markov Properties for Computational Efficiency. Norwegian
University of Science and Technology. 
\end{enumerate}

Re-read \cite{jones_application_2017}, ``Finally, prescribed burning presents a dilemma for the rational policymaker. While it can reduce fuel loads and lessen the intensity of future wildfires, it also creates immediate smoke effects and associated costs. Considering this tradeoff is important given the significant consequences of smoke exposure.''

Read Haikerwal et al \url{http://dx.doi.org/10.1080/10962247.2015.1032445}

Read Liu et al., 2015 \url{http://dx.doi.org/10.1016/j.envres.2014.10.015}

Read \cite{liu_wildfire-specific_2016}

Read \cite{kim_mutagenicity_2018} for flaming vs smoldering smoke emissions

Read Fantke et al, 2015 \cite{http://dx.doi.org/10.1007/s11367-014-0822-2}

Read Hanninen et al 2009 \cite{http://dx.doi.org/10.1038/jes.2008.31}

Read Williamson GJ, Bowman DMJS, Price OF, Henderson SB, Johnston FH \cite{williamson_transdisciplinary_2016} A transdisciplinary approach to understanding the health effects of wildfire and prescribed fire smoke regimes. Environmental Research Letters . (cited in \cite{hyde_air_2017}

read Liu et al 2015; Gan et al 2017

See \cite{vaidyanathan_developing_2018} for info about Medicaid Beneficiaries and Children's Health Insurance Program as well as Healthcare Cost and Utilization Project (HCUP)