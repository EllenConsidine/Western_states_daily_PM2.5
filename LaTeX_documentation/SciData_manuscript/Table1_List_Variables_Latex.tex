\begin{longtable}{l|l|l} \caption{Variables used in the machine learning models.} \label{tab:Table1} \\ 
\hline 
\textbf{Variables}  & \textbf{Type}  & \textbf{Source}  \\ 
 \hline 
\begin{tabular}[c]{@{}l@{}}Coordinates in degrees (Latitude \\and Longitude)\end{tabular}  & Spatial  & \begin{tabular}[c]{@{}l@{}}PM2.5 monitoring \\data\end{tabular}  \\ 
 \hline 
\begin{tabular}[c]{@{}l@{}}Count of Active Fire Points within \\each of the following radial \\buffers (25 km, 50 km, 100 \\km, and 500 km) for same day \\and up to 7 lag days for each \\radial buffer\end{tabular}  & Spatiotemporal  & \begin{tabular}[c]{@{}l@{}}MODIS Thermal Anomalies/Fire \\Daily L3 Global \\1km product\end{tabular}  \\ 
 \hline 
\begin{tabular}[c]{@{}l@{}}Binary Fire indicator (0 for no \\active fire points in any buffer \\radii or lag for given point; \\1 otherwise)\end{tabular}  & Spatiotemporal  & \begin{tabular}[c]{@{}l@{}}MODIS Thermal Anomalies/Fire \\Daily L3 Global \\1km product\end{tabular}  \\ 
 \hline 
\begin{tabular}[c]{@{}l@{}}Summed length (in meters) of arterial \\(A) and collector (C) \\roads and both (A + C) within \\100, 250, 500, and 1000 m buffer \\radii\end{tabular}  & Spatial  & \begin{tabular}[c]{@{}l@{}}National Highways Planning \\Network \end{tabular}  \\ 
 \hline 
Population Density  & Spatial  & 2010 U.S. Census  \\ 
 \hline 
AOD (unitless)  & Spatiotemporal  & \begin{tabular}[c]{@{}l@{}}MODIS Terra and Aqua \\combined Multi-angle \\Implementation of \\Atmospheric Correction \\(MAIAC) dataset\end{tabular}  \\ 
 \hline 
\begin{tabular}[c]{@{}l@{}}Planetary Boundary Layer Height \\(m)\end{tabular}  & Spatiotemporal  & NAM  \\ 
 \hline 
\begin{tabular}[c]{@{}l@{}}Temperature at 2 m above ground \\(K)\end{tabular}  & Spatiotemporal  & NAM  \\ 
 \hline 
\begin{tabular}[c]{@{}l@{}}Relative humidity at 2 m above \\ground (\%)\end{tabular}  & Spatiotemporal  & NAM  \\ 
 \hline 
\begin{tabular}[c]{@{}l@{}}Dew point temperature at 2 m above \\ground (K)\end{tabular}  & Spatiotemporal  & NAM  \\ 
 \hline 
Snow Cover (\%)  & Spatiotemporal  & NAM  \\ 
 \hline 
\begin{tabular}[c]{@{}l@{}}U-component (east/west) of wind \\at 10 m above ground (m/s)\end{tabular}  & Spatiotemporal  & NAM  \\ 
 \hline 
\begin{tabular}[c]{@{}l@{}}V-component (north/south) of wind \\at 10 m above ground (m/s)\end{tabular}  & Spatiotemporal  & NAM  \\ 
 \hline 
Mean sea level pressure (Pa)  & Spatiotemporal  & NAM  \\ 
 \hline 
Surface pressure (Pa)  & Spatiotemporal  & NAM  \\ 
 \hline 
\begin{tabular}[c]{@{}l@{}}Vertical Wind Velocity (Geometric) \\at 850 mb (m/s)\end{tabular}  & Spatiotemporal  & NAM  \\ 
 \hline 
\begin{tabular}[c]{@{}l@{}}Vertical Wind Velocity (Geometric) \\at 700 mb (m/s)\end{tabular}  & Spatiotemporal  & NAM  \\ 
 \hline 
\begin{tabular}[c]{@{}l@{}}\% of urban development within \\each of the following radial \\buffers (1 km, 5 km, and 10 \\km)\end{tabular}  & Spatial  & \begin{tabular}[c]{@{}l@{}}National Land Cover \\Database\end{tabular}  \\ 
 \hline 
\begin{tabular}[c]{@{}l@{}}Normalized Difference Vegetation \\Index (NDVI)\end{tabular}  & Spatiotemporal  & MODIS   \\ 
 \hline 
\begin{tabular}[c]{@{}l@{}}Season (Winter = December-February; \\Spring = March-May; Summer \\= \\June-August; Fall = September-November)\end{tabular}  & Temporal  & derived from date  \\ 
 \hline 
Indicator variables for state  & Spatial  & \begin{tabular}[c]{@{}l@{}}derived from latitude \\and longitude\end{tabular}  \\ 
 \hline 
Indicator variable for year  & Temporal  & derived from date  \\ 
 \hline 
Month  & Temporal  & derived from date  \\ 
 \hline 
Cosine of Day of Week  & Temporal  & derived from date  \\ 
 \hline 
Cosine of Day of Year  & Temporal  & derived from date  \\ 
 \hline 
Elevation  & Spatial  & NED  \\ 
 \hline 
Date  & Temporal  & \begin{tabular}[c]{@{}l@{}}derived from PM2.5 monitoring \\data\end{tabular}  \\ 
 \hline 
\end{longtable} 
