\begin{longtable}{l|l|l} \caption{Variables used in the machine learning models.} \label{tab:Table1} \\ 
\hline 
\textbf{Variable}  & \textbf{Type}  & \textbf{Source}  \\ 
 \hline 
Date  &  &  \\ 
 \hline 
\begin{tabular}[c]{@{}l@{}}Coordinates (Latitude \\and Longitude)\end{tabular}  & Spatial  &  \\ 
 \hline 
\begin{tabular}[c]{@{}l@{}}Active Fire Points Count (25 km, \\50 km, 100 km, and 500 km buffer \\radii; 0-7 day lags)\end{tabular}  & Spatial and Temporal  &  \\ 
 \hline 
\begin{tabular}[c]{@{}l@{}}Binary Fire indicator (0 for no \\active fire points in any buffer \\radii or lag for given point; \\1 otherwise)\end{tabular}  & Spatial and Temporal  &  \\ 
 \hline 
\begin{tabular}[c]{@{}l@{}}Summed length of arterial (A) \\and collector (C) roads within \\100, 250, 500, and 1000 m buffer \\radii, A and C separately \\and together\end{tabular}  & Spatial  & \begin{tabular}[c]{@{}l@{}}National Highways Planning Network \\\url{https://www.fhwa.dot.gov/planning/processes/tools/nhpn/index.cfm}\end{tabular}  \\ 
 \hline 
Population Density  & Spatial  &  \\ 
 \hline 
MAIAC AOD  & Spatial and Temporal  & NAM  \\ 
 \hline 
HPBL.surface  & Spatial and Temporal  & NAM  \\ 
 \hline 
TMP.2.m.above.ground  & Spatial and Temporal  & NAM  \\ 
 \hline 
RH.2.m.above.ground  & Spatial and Temporal  & NAM  \\ 
 \hline 
DPT.2.m.above.ground  & Spatial and Temporal  & NAM  \\ 
 \hline 
APCP.surface  & Spatial and Temporal  & NAM  \\ 
 \hline 
WEASD.surface  & Spatial and Temporal  & NAM  \\ 
 \hline 
SNOWC.surface  & Spatial and Temporal  & NAM  \\ 
 \hline 
UGRD.10.m.above.ground  & Spatial and Temporal  & NAM  \\ 
 \hline 
VGRD.10.m.above.ground  & Spatial and Temporal  & NAM  \\ 
 \hline 
PRMSL.mean.sea.level  & Spatial and Temporal  & NAM  \\ 
 \hline 
PRES.surface  & Spatial and Temporal  & NAM  \\ 
 \hline 
DZDT.850.mb  & Spatial and Temporal  & NAM  \\ 
 \hline 
DZDT.700.mb  & Spatial and Temporal  & NAM  \\ 
 \hline 
TimeZone  & Spatial  &  \\ 
 \hline 
\begin{tabular}[c]{@{}l@{}}National Land Cover Database (NLCD) \\(1 km, 5 km, and 10 km)\end{tabular}  & Spatial and Temporal  &  \\ 
 \hline 
NDVI  & Spatial and Temporal  &  \\ 
 \hline 
Season  & Temporal  &  \\ 
 \hline 
State  & Spatial  &  \\ 
 \hline 
Cosine of Day of Year  & Temporal  &  \\ 
 \hline 
\end{longtable} 
