\subsection{Landsat-derived burned area essential climate variable (BAECV) fire activity data}
\subsubsection*{Data Source}
\begin{itemize}[nolistsep]
\item \textbf{Contact}
\item \textbf{Citation/Link}
\item \textbf{Data (local)}
\item \textbf{Geographic Extent}
\item \textbf{Temporal Extent}
\item \textbf{Acknowledgment}
\end{itemize}
\subsubsection*{Brief Description}

We will collect data about fire detection locations, size, and fire radiative power from the Landsat-derived burned area essential climate variable (BAECV) fire activity data, 
\citep{MODISBurnArea}. 
Using GIS techniques, we will create daily clusters of fire points and use these to calculate: (1) the distance to the nearest fire cluster by day and (2) the sum of Fire Radiative Power (FRP) of the nearest clusters of fires by day as it is likely that smoke levels are higher closer to fires.  The BAECV can detect fires larger than 4 km\textsuperscript{2} and provides an estimate of the date of the fire and is available from 1984-2015. 

\subsubsection*{Notes}
\subsubsection*{File Format} .shp
\subsubsection*{Data Filtering and Processing}
\subsubsection*{Final Variable(s)}
\subsubsection*{Methods}
\begin{enumerate}
\item Navigate to \href{https://rmgsc.cr.usgs.gov/outgoing/baecv/BAECV_CONUS_v1.1_2017/}{USGS BAECV download page} and click years to download.
\end{enumerate}
\subsubsection*{Quality Control}
\subsubsection*{Script Names}
\begin{enumerate}
\item n/a
\end{enumerate}
\subsubsection*{Data File Names}
\begin{enumerate}
\item n/a
\end{enumerate}